\section{Esonero 2}
\begin{enumerate}
    \item Formulare il problema dell'interpolazione polinomiale e dimostrare l'esistenza ed unicità del polinomio di Newton.
    \item Scrivere la forma di Lagrange del polinomio interpolante le coppie di dati $(0,1),\, (2,3),\, (4,5)$.
    \item Determinare la forma di Newton del polinomio interpolante i dati $(-1, 1),\, (1, 3),\, (4, -3)$.
    \item Costruire il polinomio di Hermite interpolante i dati (della forma $(x_i, f_i, f_i')$)) $(-1, 3, 1)$ e $(1, 5, 5)$.
    \item Definire le ascisse di Chebyshev per il polinomio di grado $n$ su un generico intervallo $[a,b]$ e spiegarne l'importanza nell'ambito dell'interpolazione polinomiale.
    \item Definire una funzione di \textit{spline} di grado $m$ su una partizione $\Delta$ assegnata. Quante condizioni servono per individuarla univocamente?
    \item Scrivere il problema algebrico che definisce il polinomio di approssimazione ai minimi quadrati di grado 3, per le coppie di dati (0,2), (2,0), (2,3), (3,2), (3,3), (4,5), (5,4), e stabilire se esso esiste ed è unico.
    \item Derivare le formule di quadratura di Newton-Cotes.
\end{enumerate}

\paragraph{1.} Teorema \ref{th:esistenza_unicita_polinomio_interpolante} con dimostrazione annessa.

\paragraph{2.} $p(x)= 1\cdot L_{02}(x)+3\cdot L_{12}(x)+5 \cdot L_{22}(x),$ con i polinomi di Lagrange della forma $L_{i2}(x),\, i=0,1,2,$ dati da:
\begin{equation*}
    L_{02}(x) = \frac{(x-2)(x-4)}{(0-2)(0-4)},\quad L_{12}(x)=\frac{x(x-4)}{2(2-4)},\quad L_{22}(x)=\frac{x(x-2)}{4(4-2)}.
\end{equation*}

\paragraph{3.} $p(x)=f[-1]+ f[-1,1](x+1) + f[-1,1,4](x+1)(x-1)$, dove le differenze divise sono calcolate come segue:
\begin{center}
    \begin{tabular}{|c|c|c|c|} 
    \hline
     & 0 & 1 & 2\\
    \hline
    -1 & $\boldsymbol{f[-1]=1}$ & &   \\ 
    1 & $f[1]=3$ & $\boldsymbol{f[-1,1]=1}$ & \\ 
    4 & $f[4]=-3$ & $f[1,4]=-2$ & $\boldsymbol{f[-1, 1, 4]=-\frac{3}{5}}$ \\
    \hline
    \end{tabular}
\end{center}

\paragraph{4.} $p(x)=f[-1]+ f[-1,-1](x+1) + f[-1,-1,1](x+1)^2 +f[-1,-1,1](x+1)^2(x-1)$, dove le differenze divise sono calcolate come segue:
\begin{center}
    \begin{tabular}{|c|c|c|c|c|} 
    \hline
     & 0 & 1 & 2 & 3\\
    \hline
    -1 & $\boldsymbol{f[-1]=3}$ & & &\\ 
    -1 & $f[-1]=3$ & $\boldsymbol{f[-1,-1]=1}$ & &\\ 
    1 & $f[1]=5$ & $f[-1,1]=-1$ & $\boldsymbol{f[-1, -1, 1]=0}$ & \\
    1 & $f[1]=5$ & $f[1,1] = 5$ & $f[-1,1,1]=2$ & $\boldsymbol{f[-1,-1,1,1]=1}$\\
    \hline
    \end{tabular}
\end{center}

\paragraph{5.} Le ascisse di Chebyshev richieste sono definite come segue:
\begin{equation*}
    x_{n-i}=\frac{a+b}{2}+\frac{b-a}{2}\cos\left(\frac{2i+1}{2n+2}\right),\quad i=0,1,\hdots, n.
\end{equation*}
La loro importanza è dovuta al fatto che permettono una crescita quasi ottimale della costante di Lebesgue, $\Lambda_n\approx O(\log n)$, che è il numero di condizionamento del problema.

\paragraph{6.} $s_m(x)$ è la spline di grado $m$ su una partizione $\Delta=\{x_0<x_1<\hdots<x_n\}$ se soddisfa le seguenti proprietà:
\begin{enumerate}
    \item $s_m|_{[x_{i-1},x_i]}(x)\in\Pi_m,\; i=1,\hdots,n$;
    \item $s_m(x)\in C^{(m-1)}[x_0,x_n].$
\end{enumerate}

\noindent Sono necessarie $m+n$ condizioni indipendenti per individuare una particolare spline.

\paragraph{7.} Se il polinomio in questione è $p(x)=a_0 + a_1x + a_2x^2 + a_3x^3$, i suoi coefficienti sono individuati dal sistema lineare sovradeterminato
\begin{equation*}
    \begin{bmatrix}
        0^0 & 0^1 & 0^2 & 0^3\\
        2^0 & 2^1 & 2^2 & 2^3 \\
        2^0 & 2^1 & 2^2 & 2^3 \\
        3^0 & 3^1 & 3^2 & 3^3 \\
        3^0 & 3^1 & 3^2 & 3^3 \\
        4^0 & 4^1 & 4^2 & 4^3 \\
        5^0 & 5^1 & 5^2 & 5^3
    \end{bmatrix}
    \begin{bmatrix}
        a_0\\
        a_1\\
        a_2\\
        a_3
    \end{bmatrix} = 
    \begin{bmatrix}
        2\\
        0\\
        3\\
        2\\
        3\\
        5\\
        4
    \end{bmatrix}
\end{equation*}
nel senso dei minimi quadrati. La soluzione esiste ed è unica se la matrice dei coefficienti ha rango massimo. In questo caso la matrice ha rango massimo perché è del tipo di Vandermonde e vi sono almeno 4 ascisse distinte.

\paragraph{8.} È necessario approssimare
\begin{equation*}
    I(f)=\int_a^bf(x)dx
\end{equation*}
con l'integrale del polinomio interpolante $f(x)$ sulle ascisse equidistanti $x_i=a+ih,\, i=0,\hdots,n,\, h=\frac{b-a}{n}$, con il polinomio espresso in forma di Lagrange come
\begin{equation*}
    p(x)=\sum_{i=0}^nf_i L_{in}(x).
\end{equation*}

È ottenuto quanto segue:
\begin{equation*}
    I(f)\approx I_n(f) \equiv I(p)=\int_a^b\sum_{i=0}^nf_iL_{in}(x)dx = \sum_{i=0}^nf_i\int_a^b L_{in}(x)dx.
\end{equation*}

Posto $x=a+th$, è ottenuto che $dx=hdt,\, x_i=a+ih,\, t\in[0,n]$ e
\begin{equation*}
    \int_a^bL_{in}(x)dx=h\int_0^nL_{in}(a+th)dt = h \underbrace{\int_0^n\prod_{j=0,\, j\neq i}^n\frac{t-j}{i-j}dt}_{c_{in}}.
\end{equation*}

Pertanto,
\begin{equation*}
    I_n(f)=\frac{b-a}{n}\sum_{i=0}^nc_{in}{f_i},
\end{equation*}
che è la formula di Newton-Cotes di grado $n$.