\section{Esercitazione capitoli 4 e 5}
\subsection{A.A. 2022/23}
\begin{enumerate}
    \item  Formulare il problema dell'interpolazione polinomiale e dimostrare l'esistenza ed unicità del polinomio interpolante.
    \item Scrivere la forma di Lagrange del polinomio interpolante le coppie di dati (0, 1), (2, 3), (3, 7).
    \item Determinare la forma di Newton del polinomio interpolante i dati (0, 3), (2, 4), (3, 5). Sapendo che la derivata terza della funzione interpolanda ha norma minore o uguale a 5, dare una stima dell'errore di interpolazione nel punto $x = 4$.
    \item Derivare l'espressione dell'errore nell'interpolazione polinomiale.
    \item Costruire il polinomio di Hermite interpolante i dati,nella forma $(x_i, f_i , f'_i)$, (0, 1, 7) e (3, 4, -8).
    \item Definire la costante di Lebesgue, e spiegarne esaurientemente il significato nell'ambito dell'interpolazione polinomiale.
    \item Definire le ascisse di Chebyshev, e spiegarne il significato nell'ambito dell'interpolazione polinomiale.
    \item  Scrivere una \textit{function} Matlab che calcoli le ascisse di Chebyshev per costruire il polinomio interpolante di grado $n$ su un generico intervallo $[a, b]$.
    \item Cosa è il polinomio di migliore approssimazione di una funzione, e come questo è collegato all'errore nell'interpolazione polinomiale?
    \item Definire i polinomi di Chebyshev di prima specie, ed i corrispondenti polinomi monici. Qual è la caratteristica saliente di questi ultimi?
    \item Definire una funzione \textit{spline} di grado $m$ su una partizione $\Delta$ assegnata.
    \item Qual è l'unica funzione \textit{spline} univocamente determinata dalle condizionidi interpolazione sui nodi della partizione assegnata? Scriverne esplicitamente l'espressione.
    \item  Quante condizioni (indipendenti tra loro) servono per individuare univocamente una funzione \textit{spline}?
    \item Qual è l'unica funzione spline univocamente determinata dalle condizioni di interpolazione sui nodi della partizione assegnata? Scriverne esplicita- mente l'espressione.
    \item Quante condizioni servono per determinare univocamente una spline cu- bica interpolante? Cosa si intende per spline cubica interpolante naturale (completa/periodica/not-a-knot)?
    \item Scrivere una function Matlab che risolva efficientemente un sistema di equazioni tridiagonale (supporre che la matrice sia fattorizzabile $LU$).
    \item Definire il polinomio approssimante ai minimi quadrati. Sotto quali condizioni esso esiste ed è unico?
    \item Scrivere il problema algebrico che definisce il polinomio di approssimazione ai minimi quadrati di grado 2, per le coppie di dati (1,2), (1,2.1) (2,3), (3,4), (3,3) (4,5).
    \item Studiare il problema del condizionamento di un integrale definito.
    \item Derivare le formule di quadratura di Newton-Cotes.
    \item Calcolare $\int_0^\pi\cos{(x)}dx$ e la sua approssimazione con la formula dei trapezi e quella di Simpson, dettagliando i passaggi.
    \item Qual è l'espressione dell'errore della formula di Newton-Cotes di grado $k$? Quale quella della formula composita, supponendo che n sia multiplo di $k$?
    \item Come è possibile ottenere una stima dell'errore per la formula di Newton- Cotes di grado $k$, usata con n multiplo pari di $k$?
    \item Scrivere una function Matlab che calcoli efficientemente l'approssimazione di un integrale definito mediante la formula dei trapezi composita. Supporre che la function della funzione integranda accetti input vettoriali.
\end{enumerate}

\paragraph{2.}
\begin{equation*}
    p(x)=\sum_{i=0}^n f_iL_{in}(x),\quad L_{in}(x)=\prod_{j=0,\, j\neq i}^n\frac{x-x_j}{x_i-x_j},\quad i=0,\hdots,\, n.
\end{equation*}
Nel nostro caso:
$\begin{cases}
    n=2,\\
    x_0=0,\, x_1=2,\, x_2 = 3,\\
    f_0 = 1,\, f_1=3,\, f_2=7.
\end{cases}$

\noindent Pertanto:
\begin{equation*}
    p(x) = 1\cdot L_{02}(x)+3\cdot L_{12}(x) + 7\cdot L_{22}(x),
\end{equation*}
con
\begin{equation*}
    \begin{matrix}
        L_{02}(x) &=& \frac{(x-2)(x-3)}{(0-2)(0-3)} &=& \frac{(x-2)(x-3)}{6};\\
        L_{12}(x) &=& \frac{(x-0)(x-3)}{(2-0)(2-3)} &=& -\frac{1}{2}x(x-3);\\
        L_{22}(x) &=& \frac{(x-0)(x-2)}{(3-0)(3-2)} &=& \frac{x(x-2)}{3}.
    \end{matrix}
\end{equation*}
È possibile che inserisca un esercizio trabocchetto del tipo: calcolare polinomio interpolante, una funzione (polinomio) di grado 3, fornendo 18 coppie di punti. Questo significa che è ricercato il polinomio di grado al più 17. È chiaro che con il calcolo della funzione, questo sarà di una funzione minimale.

\paragraph{3.} In genere, si ha che
\begin{equation*}
    p(x) = \sum_{i=0}^n f[x_0,\hdots,\, x_i]\,\omega_i(x),
\end{equation*}
dove $f[x_0,\hdots, x_i]$ è la differenza divisa di $f$ sulle ascisse in argomento, e
\begin{equation*}
    \underset{\footnotemark}{w_i(x)}=\prod_{k=0}^{i-1}(x-x_k),\, i=0,\hdots, n.
\end{equation*}

\noindent Nel nostro caso:
$\begin{cases}
    n=2,\\
    x_0=0,\, x_1=2,\, x_2 = 3,\\
    f_0 = 3,\, f_1=4,\, f_2=5.
\end{cases}$

\noindent Pertanto:
\begin{itemize}
    \item[a)] $w_0(x)\equiv 1,\quad w_1(x)=x,\, w_2(x)=x(x-2),$
    \item[b)]
    \begin{tabular}{c|ccc}
         & 0 & 1 & 2 \\
         \hline
        $x_0=0$ & $\boldsymbol{f[x_0]=3}$\\
        $x_1=2$ & $f[x_1]=4$ & $\boldsymbol{f[x_0, x_1]=\frac{1}{2}}$\\
        $x_2=3$ & $f[x_2]=5$ & $f[x_1, x_2]=1$ & $\boldsymbol{f[x_0, x_1, x_2]=\frac{1}{6}}$
    \end{tabular}
    
    \noindent Concludiamo che:
    \begin{equation*}
        p(x)=3+\frac{1}{2}x+\frac{1}{6}x(x-2).\quad \text{[Fine prima parte del quesito]}
    \end{equation*}
\end{itemize}

\noindent [Risposta alla seconda parte del quesito:] Sappiamo che l'errore di interpolazione è dato da:
\begin{equation*}
    e(x)=f[x_0,x_1,x_2,x]\,\omega_3(x),\quad \omega_3(x)=x(x-2)(x-3).
\end{equation*}

\noindent Inoltre,
\begin{equation*}
    f[x_0,x_1,x_2,x]=\frac{f^{(3)}(\xi_x)}{3!},
\end{equation*}
per un opportuno $\xi_x\in I(x,0,2,3)$. Poiché $\overset{\footnotemark}{||f^{(3)}||}\leq 5$, abbiamo che:
\begin{equation*}
    |e(4)|\overset{\footnotemark}{\leq}\frac{||f^{(3)}||}{3!}|\omega_3(4)|\leq\frac{\overset{\footnotemark}{5}}{\equalto{6}{3!}}\cdot \equalto{8}{|\omega_3(4)|}=\frac{20}{3}.
\end{equation*}
\addtocounter{footnote}{-2}
\footnotetext{Massimo del valore assoluto.}

\stepcounter{footnote}
\footnotetext{Dovuto alla norma, la quale è il massimo del valore assoluto.}

\stepcounter{footnote}
\footnotetext{Dovuto all'ipotesi $||f^{(3)}||\leq 5$.}

\paragraph{5.} In generale, si ha che:
\begin{equation*}
    \begin{matrix}
        p_H(x)&=&f[x_0]+(x-x_0)f[x_0,x_0]+(x-x_0)^2f[x_0,x_0,x_1] + \hdots +\\
        && + (x-x_0)^2\cdot\hdots\cdot (x-x_{n-1})^2(x-x_n)f[x_0,x_0,\hdots,x_n,x_n].
    \end{matrix}
\end{equation*}

\noindent Nel nostro caso:
$\begin{cases}
    n=1,\, x_0=0,\, x_1=3,\\
    f(x_0)=1,\, f(x_1)=4,\\
    f'(x_0)=7,\, f'(x_1)=-8.
\end{cases}$

\noindent Pertanto:
\begin{equation*}
    (x-x_0)=x,\, (x-x_0)^2 = x^2,\, (x-x_0)^2(x-x_1) = x^2(x-3).
\end{equation*}
Inoltre: \footnote{I coefficienti in grassetto sono necessari.}
\begin{center}
    \begin{tabular}{c|cccc}
         & 0 & 1 & 2 & 3\\
         \hline
         $x_0 = 0$ & $\boldsymbol{f[x_0]=1}$\\ 
         $x_0 = 0$ & $f[x_0]=1$ & $\boldsymbol{f[x_0,x_0]=f'(x_0)=7}$\\
         $x_1=3$ & $f[x_1]=4$ & $f[x_0,x_1]=1$ & $\boldsymbol{f[x_0,x_0,x_1]=-2}$\\
         $x_1=3$ & $f[x_1]=4$ & $f[x_1,x_1]=f'(x_1)=-8$ & $f[x_0,x_1,x_1]=-3$ & $\boldsymbol{f[x_0,x_0,x_1,x_1]=-\frac{1}{3}}$
    \end{tabular}
\end{center}
Pertanto,
\begin{equation*}
    P_H(x)=1+7x-2x^2-\frac{1}{3}x^2(x-3).
\end{equation*}

\paragraph{8.} Algoritmo \ref{alg:cheby_esercitazione}.
\begin{algorithm}
\caption{Implementazione esercizio 8.}\label{alg:cheby_esercitazione}
    \begin{lstlisting}[style=Matlab-editor]
    function x = cheby(n, a, b)
    %   
    %  x = cheby(n, a, b)
    %
    %   Ascisse di Chebyshev:
    %
    %   n   : grado del polinomio;
    %   a,b : estremi dell'intervallo.
    %   
    if nargin < 3, error('numero argomenti insufficienti'), end
    if n ~= fix(n) || n <= 0, error('grado non corretto'), end
    x = cos((2*(0:n)+1)*pi/(2*(n+1))); %ascisse su [-1, 1]. Passo indispensabile. Se il polinomio e' di grado n, allora il numero di ascisse e' n+1.
    x = (a+b)/2 + x*(b-a)/2;
    return
    \end{lstlisting}
\end{algorithm}

\paragraph{9.} Data una funzione $C[a,b]$, si definisce polinomio di migliore approssimazione di grado $n$ di $f$ su $[a,b]$:
\begin{equation*}
    p^* = \arg\underset{p\in\Pi_n}{\min}||f-p||.
\end{equation*}

\noindent È noto che, se $p(x)\in\Pi_n$ è un polinomio interpolante $f(x)$ su $n+1$ ascisse in $[a,b]$, allora:
\begin{enumerate}
    \item $||f-p||\leq (1+\Lambda_n)||f-p||$, essendo $\Lambda_n$ la costante di Lebesgue definita sulle ascisse di interpolazione;
    \item $||f-p^*||\leq \alpha\cdot\omega\left(f;\frac{b-a}{n}\right)$, dove $\alpha$ è indipendente da $n$ e $\omega\left(f;\frac{b-a}{n}\right)$ è il modulo di continuità di $f(x)$.\\
\end{enumerate}
Da 1. e 2. si deduce il Teorema di Jackson:
\begin{equation*}
    ||f-p||\leq\alpha(1+\Lambda_n)\,\omega\left(f;\frac{b-a}{n}\right).
\end{equation*}

\paragraph{10.} I polinomi di Chebyshev di prima specie sono definiti dall'equazione di ricorrenza:
\begin{equation*}
    \begin{cases}
        T_0(x) \equiv 1;\\
        T_1(x) = x;\\
        T_{k+1}(x) = 2x\cdot T_k(x)-T_{k-1}(x),\;\; k\geq 1
    \end{cases}\quad x\in[-1,1].
\end{equation*}
Sappiamo che:
\begin{enumerate}
    \item $||T_k||=1,\quad\forall k\geq 0$;
    \item Il coefficiente principale di $T_k(x)$ è $2^{k-1},\, \forall k\geq 1$;
    \item $\widehat T_k(x)=
    \begin{cases}
    T_0(x), &k=0,\\
    2^{1-k}T_k(x), &k\geq 1,
    \end{cases}$\\
    è una famiglia di polinomi monici di grado $k,\,\forall k\geq 0$;
    \item $\forall k\geq 1:||\widehat T_k||=2^{1-k}=\underset{\underset{\footnotemark}{p\in\Pi_k}}{\min}||p||$.
\end{enumerate}
\footnotetext{$p$ è un polinomio monico.}

\paragraph{11.} Assegnata la partizione $\Delta = \{x_0<x_1<\hdots<x_n\}$ diremo che $s_m(x)$ è una spline di grado $m$ su $\Delta$, se:
\begin{enumerate}
    \item $\forall i=1,\hdots,n : s_m|_{[x_{i-1},x_i]}(x)\in\Pi_m$;
    \item $s_m(x)\in C^{(m-1)}[x_0,x_n]$.
\end{enumerate}
[Specifica aggiuntiva:] La condizione 2. significa richiedere:
\begin{equation*}
    \begin{matrix}
        s_m^{(j)}|_{[x_{i-1},x_i]}(x_i)= s_m^{(j)}|_{[x_i,x_{i+1}]}(x_i),\\
        j=0,\hdots,m-1,\; i=1,,\hdots,n-1.
    \end{matrix}
\end{equation*}

\noindent [Aggiunta:] Se data una funzione $f(x)$, si ha $s_m(x_i)=f(x_i),\, i=0,\hdots,n$, allora $s_m(x)$ è una spline di grado $m$ interpolante $f(x)$ sulla partizione $\Delta$.

\paragraph{13.} Per determinare univocamente una spline di grado $m$ sulla partizione $\Delta=\{x_0<x_1<\hdots<x_n\}$, occorrono $m+n$ condizioni. Le condizioni di interpolazione sono, evidentemente, $n+1$.

\noindent Pertanto, esse individuano univocamente la spline lineare interpolante:
\begin{equation*}
    s_1(x_i)=f_i,\quad i=0,\hdots, n.
\end{equation*}

\noindent Trattandosi della spezzata congiungente i punti $(x_i,\, f_i),\, i=0,\hdots, n$, la sua espressione sarà:
\begin{equation*}
    s_1(x)=\frac{f_i(x-x_{i-1})+f_{i-1}(x_i-x)}{x_i-x_{i-1}},\quad x\in[x_{i-1, x_i}],\, i = 1, \hdots, n.
\end{equation*}

\paragraph{16.} Si tratta di determinare i coefficienti del polinomio
\begin{equation*}
    p(x)=\sum_{k=0}^m a_kx^k\in\Pi_m,
\end{equation*}
che meglio approssima le coppie di dati $(x_i,f_i),\, i=1,\hdots, n,\, \boldsymbol{n>m+1}$. Questo significa che il valore $f_i$ è approssimato con
\begin{equation*}
    f_i\approx\sum_{k=0}^ma_kx_i^k,\quad i=1,\hdots, n.
\end{equation*}
Scritto in forma vettoriale, abbiamo:
\begin{equation*}
    \begin{bmatrix}
        x_1^0&\hdots &x_1^m\\
        \vdots && \vdots\\
        x_n^0& \hdots &x_n^m
    \end{bmatrix}
    \begin{bmatrix}
        a_0\\
        \vdots\\
        a_m
    \end{bmatrix}=
    \begin{bmatrix}
        f_1\\
        \vdots\\
        f_m
    \end{bmatrix},
\end{equation*}
ovvero un sistema lineare sovradimesionato. È noto che questo ammette soluzione, mediante la fattorizzazionee $QR$ della matrice dei coefficienti, e questa è unica, s.se la matrice ha rango massimo (ovvero $m+1$). Questo significa che \textbf{almeno $\boldsymbol{m+1}$ delle ascisse $\boldsymbol{\{x_i\}}$ devono essere tra loro distinte.}

\paragraph{17.} Preliminarmente, osserviamo che il problema ammette soluzione, e questa è unica, poiché abbiamo 4 ascisse distinte e $4>m+1=3$. Quindi:
\begin{equation*}
    \begin{bmatrix}
        1^0 & 1^1 & 1^2\\
        1^0 & 1^1 & 1^2\\
        2^0 & 2^1 & 2^2\\
        3^0 & 3^1 & 3^2\\
        3^0 & 3^1 & 3^2\\
        4^0 & 4^1 & 4^2
    \end{bmatrix}
    \begin{bmatrix}
        a_0\\
        a_1\\
        a_2
    \end{bmatrix}=
    \begin{bmatrix}
        2\\
        2.1\\
        3\\
        4\\
        3\\
        5
    \end{bmatrix}
\end{equation*}
è il problema algebrico che definisce i coefficienti del polinomio di approssimazione ai minimi quadrati:
\begin{equation*}
    \boldsymbol{p(x)=a_0+a_1x+a_2x^2}.
\end{equation*}

\paragraph{19.} Il problema è quello di ottenere una approssimazione dell'integrale definito
\begin{equation*}
    I(f)=\int_a^b f(x)dx.
\end{equation*}
Questa si ottiene come l'integrale del polinomio, di grado $n$, interpolante $f(x)$ sulle ascisse, equidistanti:
\begin{equation*}
    x_ia+ih,\quad i=0,\hdots, n,\quad h=\frac{b-a}{n}.
\end{equation*}
Pertanto, $p(x_i)=f_i,\, i=0,\hdots,n$, e
\begin{equation*}
    I(p)=\int_a^b p(x)dx\equiv I_n(f),
\end{equation*}
che è la formula di Newton-Cotes di grado $n$. Otteniamo la sua espressione, utilizzando la formula di Lagrange del polinomio interpolante:
\begin{equation*}
    p(x)=\sum_{i=0}^n f_iL_{in}(x),
\end{equation*}
con
\begin{equation*}
    L_{in}(x)=\prod_{j=0, j\neq i}^n \frac{x-x_j}{x_i-x_j},\quad i=0,\hdots, n.
\end{equation*}

\noindent Osserviamo che, definendo la trasformazione $x=a+th$, allora:
\begin{enumerate}
    \item $x\in[a,b]\iff t\in[0,n]$;
    \item $x_i=a+ih,\, i=0,\hdots,n$;
    \item $dx=\boldsymbol h dt$;
    \item $L_{in}(x) = L_{in}(a+th) = \prod_{j=0,j\neq i}^n\frac{(\cancel{a}+t\cancel{h})-(\cancel{a}+i\cancel{h})}{(\cancel{a}+i\cancel{h})-(\cancel{a}+j\cancel{h})} = \prod_{j=0,j\neq i}^n\frac{t-j}{i-j}\equiv \widehat L_{in}(t),\, i=0,\hdots, n.$
\end{enumerate}
Pertanto:
\begin{equation*}
    \begin{matrix}
        I_n(f) &=& \int_a^b p(x) dx &=& \int_a^b\sum_{i=0}^n f_iL_{in}(x)dx\\
        &\overset{\boldsymbol{x=a+th}}{=}& h\int_0^h\sum_{i=0}^nf_i \widehat L_{in}(t)dt &=& h \sum_{i=0}^n fi \int_0^n \widehat L_{in}(t)dt &=& \frac{b-a}{n} \sum_{i=0}^n f_i \underbrace{\int_0^n\prod_{j=0, j\neq i}^n\frac{t-j}{i-j}dt}_{c_{in}},
    \end{matrix}
\end{equation*}
che è la forma finale della formula.

\paragraph{23.} È noto che, denotando con $I(f)=\int_a^b f(x) dx$ l'integrale da approssimare, e con $I_k^{(n)}(f)$ l'approssimazione fornita dalla composita di grado $k$:
\begin{equation*}
    I(f) - I_k^{(n)}(f)-\nu_kf^{(k+\mu)}(\boldsymbol\xi)\frac{(b-a)}{k}\left(\frac{b-a}{n}\right)^{k+\mu},
\end{equation*}
con
$\mu =\begin{cases}
     1, & k \text{ dispari,}\\
     2, & k \text{ pari,}
\end{cases}$

\noindent il coefficiente $\nu_k$ dipendente solo da $k$, e $\xi\in[a,b]$, opportuno.

\noindent Utilizzando solo le valutazioni funzionali sulle ascisse $x_{2i},\, i=0,\hdots, \frac{n}{2}$, con $n=2\,k\,m,\, m\in\mathbb N$, otteniamo $I_k^{(\frac{n}{2})}(f)$, tali che:
\begin{equation*}
    I(f) - I_k^{(\frac{n}{2})}(f)=\nu_k f^{n+\mu}(\boldsymbol{\widehat\xi})\frac{b-a}{k}\left(\frac{b-a}{n/2}\right)^{k+\mu},
\end{equation*}
con $\widehat\xi\in [a,b]$, opportuno. Utilizzando l'approssimazione $\boldsymbol{\xi\approx\widehat\xi}$, sottraendo membro a membro, otteniamo:
\begin{equation*}
    I_k^{(n)}(f)-I_k^{(\frac{n}{2})}(f)\approx\boldsymbol{\nu_kf^{(n+\mu)}(\xi)\frac{b-a}{k}\cdot \left(\frac{b-a}{n}\right)^{k+\mu}} \left(2^{k+\mu}-1\right) = \left[\boldsymbol{I(f)-I_k^{(n)}(f)}\right]\left(2^{k+\mu}-1\right)
\end{equation*}

\noindent Pertanto, otteniamo la stima
\begin{equation*}
    I(f)-I_k^{(n)}(f) \approx \frac{I_k^{(n)}(f)-I_k^{(\frac{n}{2})}(f)}{2^{k+\mu}-1}.
\end{equation*}

\paragraph{24.} Vedere l'Algoritmo \ref{alg:trapec}
\begin{algorithm}
\caption{Implementazione esercizio 24.}\label{alg:trapec}
    \begin{lstlisting}[style=Matlab-editor]
    function If = trapec(fun, a, b, n)
    %   
    %   If = trapec(fun, a, b, n)
    %
    %   Calcolo della formula composita dei trapezi.
    %
    %  Input:
    %   fun - identificatore function funzione intreganda (deve accettare input vettoriali);
    %   a,b - estremi dell'intervallo di integrazione;
    %   n   - numero sottointervallli (default=1).
    %  Output:
    %   If - stima ottenuta.
    %
    if nargin < 3
        error('numero argomenti insufficiente')
    elseif nargin == 3
        n = 1;
    elseif n <= 0 || n~= fix(n)
        error('numero di sottointervalli non valido')
    end
    x = linspace(a, b, n+1);
    h = (b-a)/n;
    f = feval(fun, x);
    If = h * (sum(f) - (f(1) + f(n+1))/2);
    return
    end
    \end{lstlisting}
\end{algorithm}
